\documentclass[a4paper,twocolumn,10pt]{article}
\usepackage[utf8]{inputenc}
\usepackage{amsmath} % align environment
\usepackage{mathptmx} % times roman, including math
\usepackage[hyphens]{url}
\usepackage{doi}
\usepackage{hyperref}
\usepackage[numbers,sort]{natbib}
\hyphenation{da-ta-cen-ter da-ta-cen-ters}
\frenchspacing

% Placeholder character like \textvisiblespace, but works in math mode
\newcommand\placeholder{%
  \makebox[0.7em]{%
    \kern.07em
    \vrule height.3ex
    \hrulefill
    \vrule height.3ex
    \kern.07em
  }%
}

\begin{document}
\sloppy
\title{Composing Data Structures for Collaborative Document Editing}
\author{}
\maketitle

\subsection*{Abstract}

\section{Introduction}

CRDTs~\cite{Shapiro:2011wy,Roh:2011dw}

\section{Operational Semantics}

We assume that each peer has a unique identifier (for example, the hash of its public key). Whenever an edit to a document is made at one of the peers, we generate a unique identifier for that edit operation using Lamport timestamps~\cite{Lamport:1978jq}.

A Lamport timestamp is a pair $(c, p)$ where $p$ is the unique identifier of the peer on which the edit is made. $c$ is a counter that is stored at each peer and incremented for every operation. Whenever an operation is sent to other peers, the Lamport timestamp $(c, p)$ for that operation is included in the message.

If a peer receives an operation with a counter value $c$ that is greater than the locally stored counter value, the local counter is increased to the value of the incoming counter. This ensures that if operation $o_1$ happened before $o_2$ (that is, the peer that generated $o_2$ had received and processed $o_1$ before $o_2$ was generated), then $o_2$ must have a greater counter value than $o_1$. Only concurrent operations can have equal counter values.

We can thus define a total ordering $<$ for Lamport timestamps:
$$(c_1, p_1) < (c_2, p_2) \;\text{ iff }\; (c_1 < c_2) \vee (c_1 = c_2 \wedge p_1 < p_2).$$
If one operation happened before another, this ordering is consistent with causality (the earlier operation has a lower timesetamp). If two operations are concurrent, their order according to $<$ is arbitrary but deterministic.

\subsection{Ordered List Operations}

The \emph{Replicated Growable Array} (RGA) datatype was originally defined by Roh et al.\ using pseudocode~\cite{Roh:2011dw}. In this section we present the algorithm in an alternative form which is more amenable to correctness proofs.

Each peer that maintains a replica of an RGA stores a copy of its state $A$, which is a partial function from $\mathit{id}$ to pairs of $(\mathit{value}, \mathit{next})$. The parameter $\mathit{id}$ is either a Lamport timestamp or the special symbol $\mathsf{head}$, indicating the head of the list. $\mathit{next}$ is either a Lamport timestamp or the symbol $\mathsf{tail}$, denoting the end of the list. $\mathit{value}$ is any value of the datatype of list elements, or the special value $\bot$, indicating the absence of a value.

The empty list $A_\emptyset$ is represented as a function whose domain is the single value $\mathsf{head}$:
$$ A_\emptyset = \{\mathsf{head} \mapsto (\bot, \mathsf{tail})\}. $$

We use the notation $A[\,a \mapsto (b, c)\,]$ to denote a function identical to $A$, except that $A(a)=(b,c)$.

A list element is created with an $\mathsf{insert}$ operation. Since each operation has a Lamport timestamp, a list element is uniquely identified by its timestamp, which remains immutable for the lifetime of the document. We call that Lamport timestamp the ID of the list element.

The operation $\mathsf{insert}(\mathit{id}, \mathit{prev}, v)$ is an instruction to insert the value $v$ into a list at a position following the existing list element with ID $\mathit{prev}$. An insertion at the head of the list is expressed as $\mathsf{insert}(\mathit{id}, \mathsf{head}, v)$. The $\mathit{id}$ is the unique ID of this operation.

When the operation is applied to the list state $A$, it produces a modified list state $A'$ as follows:
\begin{align*}
A' &= \mathrm{apply}(A, \mathsf{insert}(\mathit{id}, \mathit{prev}, v)) \\ &=
\begin{cases}
\quad \mathrm{apply}(A, \mathsf{insert}(\mathit{id}, n, v)) \\
    \qquad\quad\text{if }\; A(\mathit{prev}) = (\placeholder, n) \;\wedge\; \mathit{id} < n \\
\quad A[\,\mathit{prev} \mapsto (v_p, \mathit{id}),\; \mathit{id} \mapsto (v, n)\,] \\
    \qquad\quad\text{if }\; A(\mathit{prev}) = (v_p, n) \;\wedge\; n < \mathit{id}
\end{cases}
\end{align*}
where $<$ is the total order on Lamport timestamps, with the additional requirement that $\mathsf{tail} < (c, p)$ for any Lamport timestamp $(c, p)$.

Applying an $\mathsf{insert}$ operation is like inserting an element into a linked list, except that the function first skips over list elements with an ID greater than the ID of the new element being inserted. This has the effect of deterministically ordering concurrent insertions made at the same position of the list. This property is proved formally below.

The operation $\mathsf{delete}(\mathit{id})$ is an instruction to delete the element with ID $\mathit{id}$ from the list. The operation has the following semantics:
\begin{align*}
A' &= \mathrm{apply}(A, \mathsf{delete}(\mathit{id})) \\ &=
A[\,\mathit{id} \mapsto (\bot, n)\,]
\quad\text{if }\; A(\mathit{id}) = (\placeholder, n)
\end{align*}

The user view of a list A iterates over the list elements from $\mathsf{head}$ to $\mathsf{tail}$, skipping any deleted elements (tombstones):
\begin{align*}
\mathrm{view}(A) &= \mathrm{itr}(A, \mathsf{head}) \\
\mathrm{itr}(A, \mathit{id}) &= \begin{cases}
    [v] \mathbin{::} \mathrm{itr}(A, n) & \text{if }\; A(\mathit{id}) = (v, n) \;\wedge\; v \neq \bot \\
                     \mathrm{itr}(A, n) & \text{if }\; A(\mathit{id}) = (v, n) \;\wedge\; v = \bot \\
    [\,] & \text{if }\; \mathit{id} = \mathsf{tail}
\end{cases}
\end{align*}

\subsection{Convergence}

We now prove the convergence property, also known as \emph{eventual consistency}, for this data structure.

\textbf{Theorem.} If any two peers process the same set of $\mathsf{insert}$ and $\mathsf{delete}$ operations, in any causally consistent order (but not necessarily the same order), then their final states are identical.

\textbf{Proof.} Each peer processes operations in some sequential order, and the state of a peer is modified only by operations. Therefore, we can derive the state of peer $p$ from the history of operations $H_p=o_1 \dots o_n$ applied at that peer, processed in the order of the history:
$$ A_{H_p} = \mathrm{apply}(\mathrm{apply}(\dots \mathrm{apply}(A_\emptyset, o_1), \dots, o_{n-1}), o_n). $$

We can prove the theorem by induction over the length of the history $n$.

\emph{Base case:} A history of $n=0$ operations describes an empty document. An empty document is deterministically defined, so any two peers that have not executed any operations are by definition in the same state.

\emph{Induction step:} Given causal histories $H_1$ and $H_2$ of length $n$, such that $H_1=o_1 \dots o_n$ and $H_2$ is a permutation of $H_1$, and such that applying $H_1$ results in the same state as applying $H_2$, we can construct new histories of length $n+1$ by inserting an operation $o_{n+1}$ at any causally ready position in either $H_1$ or $H_2$. We must then show that for all of the histories constructed this way, applying the operations results in the same final state.

We show the induction step case by case, considering each type of operation $o_{n+1}$ that may be inserted.

\subsubsection{Commutativity of deletion}

First consider the case that $o_{n+1}=\mathsf{delete}(\mathit{id})$. A list element can only be deleted if it exists in the list, so by causality, there must exist an operation $o_i=\mathsf{insert}(\mathit{id}, \placeholder, \placeholder)$ at a prior point in the history. Moreover, the Lamport timestamp used as $\mathit{id}$ is unique, so there must be exactly one such $\mathsf{insert}$ operation, and there cannot be any $\mathsf{insert}$ operation for the same $\mathit{id}$ at a point after $o_{n+1}$ in the history.

Note that applying $\mathsf{delete}(\mathit{id})$ modifies only the mapping for $A(\mathit{id})$, and otherwise leaves the state $A$ unchanged, so it is trivially commutative with respect to any operation that does not modify or depend on $A(\mathit{id})$.

The first kind of operation that can modify the same $A(\mathit{id})$ is a $\mathsf{delete}$ operation for the same list element $\mathit{id}$. As the rule for applying this operation is idempotent, the presence of any other $\mathsf{delete}(\mathit{id})$ operations before or after $o_{n+1}$ in the history does not have any effect on the final state.

We showed that an $\mathsf{insert}(\mathit{id}, \placeholder, \placeholder)$ operation can only occur prior to $\mathsf{delete}(\mathit{id})$. Thus, the only other kind of operation that can depend on or modify $A(\mathit{id})$ is an $\mathsf{insert}(\placeholder, \mathit{id}, \placeholder)$ operation, or a recursive call to the $\mathrm{apply}$ function that uses such an operation internally. Note that when applying such an $\mathsf{insert}$ operation, the queries for $A(\mathit{prev})$ ignore the $\mathit{value}$ element of the tuple and leave it unchanged, and only examine and modify the $\mathit{next}$ element of the tuple. By contrast, applying $\mathsf{delete}(\mathit{id})$ modifies only the $\mathit{value}$ element of $A(\mathit{id})$, and leaves the $\mathit{next}$ element unchanged.

Thus, if $o_{n+1}=\mathsf{delete}(\mathit{id})$ is inserted at any causally ready point in a history, the final state does not depend on the insertion point, because the inserted operation does not interact with any prior or following operations.

\subsubsection{Commutativity of insertion}

Now consider the case that $o_{n+1}=\mathsf{insert}(\mathit{id}, \mathit{prev}, v)$.

{\footnotesize
\bibliographystyle{plainnat}
\bibliography{references}{}}
\end{document}
